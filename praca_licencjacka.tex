\documentclass[a4paper,12pt,oneside]{book}

% pakiety
\usepackage{polski}
%\usepackage[utf8]{inputenc} % nie można używać as per 
			     % https://tex.stackexchange.com/a/480069/177956
\usepackage{fancyhdr} % nagłówki i stopki
\usepackage{indentfirst} % WAŻNE, MA BYĆ!
\usepackage{graphicx} % to do wstawiania rysunków
\usepackage{amsmath} % to do dodatkowych symboli, przydatne
\usepackage[pdftex,
            left=1in,right=1in,
            top=1in,bottom=1in]{geometry} % marginesy
\usepackage{amssymb} % to też do dodatkowych symboli, też przydatne
\usepackage{pdfpages}
\usepackage{lipsum}
\usepackage{multirow}
\usepackage{listings}
\usepackage{caption}
\usepackage{booktabs}
\usepackage{subcaption}
\usepackage{xcolor}
\graphicspath{ {./img/} }
\DeclareCaptionType{code}[Listing][Spis listingów] 

\definecolor{codegreen}{rgb}{0,0.6,0}
\definecolor{codegray}{rgb}{0.5,0.5,0.5}
\definecolor{codepurple}{rgb}{0.58,0,0.82}
\definecolor{backcolour}{rgb}{0.95,0.95,0.92}

\lstset{
	backgroundcolor=\color{backcolour},   
	commentstyle=\color{codegreen},
	keywordstyle=\color{magenta},
	numberstyle=\tiny\color{codegray},
	stringstyle=\color{codepurple},
	basicstyle=\footnotesize,
	breakatwhitespace=false,         
	breaklines=true,                 
	captionpos=b,                    
	keepspaces=true,                 
	numbers=left,                    
	numbersep=5pt,                  
	showspaces=false,                
	showstringspaces=false,
	showtabs=false,                  
	tabsize=2,
	float=h
}

% definicje nagłówków i stopek
\pagestyle{fancy}
\renewcommand{\chaptermark}[1]{\markboth{#1}{}}
\renewcommand{\sectionmark}[1]{\markright{\thesection\ #1}}
% komenda która sprawi że nie numerowany chapter (chapter*) jest dodany do TOC
\newcommand\chap[1]{%
  \chapter*{#1}%
  \addcontentsline{toc}{chapter}{#1}}
\fancyhf{}
\fancyhead[LE,RO]{\footnotesize\bfseries\thepage}
\fancyhead[LO]{\footnotesize\rightmark}
\fancyhead[RE]{\footnotesize\leftmark}
\renewcommand{\headrulewidth}{0.5pt}
\renewcommand{\footrulewidth}{0pt}
\addtolength{\headheight}{1.5pt}
\fancypagestyle{plain}{\fancyhead{}\cfoot{\footnotesize\thepage}\renewcommand{\headrulewidth}{0pt}}


% interlinia
\linespread{1.25}


% treść
\begin{document}
% strona tytułowa
\sloppy
\thispagestyle{empty}
\includepdf{strona_tytulowa}
\newpage{}

\thispagestyle{empty}
\newpage{}

% spis treści
\tableofcontents{}
\newpage

% pusta strona - narazie zakomentowana
%\thispagestyle{empty}
%\
%\newpage

\chap{Wstęp}
Jedną z większych dziedzin uczenia maszynowego jest uczenie przez wzmacnianie (ang.
\textit{reinforcement learning}) W odróżnieniu od zarówno uczenia nadzorowanego i 
nienadzorowanego nie potrzebujemy w tym przypadku żadnych gotowych danych wejściowych i
wyjściowych. Zamiast tego, algorytm pozyskuje dane na bierząco ze środowiska do, którego
jest zastosowany. Dzięki temu, że algorytmy uczenia przez wzmacnianie nie mają tego
ograniczenia możemy zastosować je do problemów takich jak gra na giełdzie\cite{
trading_reinforcement}, czy nauka grania w gry, w swojej pracy
skupię się na tym drugim.

Celem pracy jest zaimplementowanie algorytmu uczenia przez wzmacnianie --
Q-Learning oraz zoptymalizowaniu go tak by po zastosowaniu go do własnoręcznie
zaimplementowanej gry był w stanie osiągnąć w niej możliwie najwyższy wynik.

Na początku przedstawię jaką grę wybrałem, opiszę jej zasady oraz przedstawię
szczegóły jej implementacji. Następnie  przybliżę zagadnienie uczenia przez
wzmacnianie oraz algorytmu Q\dywiz learning. Pod koniec pokażę jak
zaimplementowałem wspomniany algorytm, oraz testy po optymalizacjach algorytmu,
które zastosowałem.
\newpage{}

\chapter{Gra}
Problem, który postawiłem przed Q-learningiem to gra z 2013 roku ``Flappy Bird'' autorstwa wietnamskiego developera Dong Nguyena\cite{flappy_bird_author}.
\section{Zasady gry}
\begin{figure}[h]
\centering
\includegraphics[scale=0.65]{flappy_bird.png}
\caption{Zrzut ekratu z gry}
\end{figure}



\addcontentsline{toc}{chapter}{Bibliografia}
\bibliographystyle{IEEEtran}
\bibliography{praca_licencjacka}

\end{document}
